\documentclass{article}

\usepackage[T1,T2A]{fontenc}
\usepackage[utf8]{inputenc}
\usepackage[english,russian]{babel}

\title{Численные методы}
\author{affeeal}
\date{\today}

\begin{document}

\maketitle

\pagebreak

\tableofcontents

\pagebreak

\section{Формат отчётов}

\begin{itemize}
  \item Титульная страница.
  \item Постановка задачи.
  \item Основные теоретические сведения.
  \item Реализация (листинг).
  \item Результаты.
\end{itemize}

\section{Лекция от 12.04.2024}

\subsection{Введение}

\textit{Численные методы} --- математические методы, предполагающие получение
приближённого решения поставленной задачи. Все методы, относящиеся к этому
классу имеют методологическую погрешность. Альтернативой выступают
\textit{точные методы}, не имеющие методологической погрешности.

\textsc{Пример}. Решение СЛАУ методом Гаусса --- точный метод, методом Зейделя
(Якоби) --- приближённый метод.

\textit{Вычислительные методы} --- совокупность численных методов и точных
методов, реализованных на ЦВМ. При решении практических задач на ЦВМ применяются
вычислительные методы, в силу чего задачи также называются
\textit{вычислительными}. При решении вычислительной задачи возникают следующие
виды погрешности:

\begin{enumerate}
  \item \textit{Инструментальная} --- получена при измерении входных данных как
    вручную, так и автоматически. Бывает устранимой и неустранимой.

  \item \textit{Методологическая} --- погрешность численного метода.

  \item \textit{Вычислительная} --- обусловлена ограниченностью разрядной
    сетки и способами округления. Различают округление усечением и дополнением.
\end{enumerate}

Точные методы не имеют методологической погрешности, численные --- имеют.

\subsection{Классификация численных методов}

\begin{enumerate}
  \item \textit{Прямые (точные) методы}. Если разница между решением и
    результатом расчёта нулевая, то решение точное. Иначе --- приближённое.

  \item \textit{Методы эквивалентных преобразований}. Исходная задача заменяется
    эквивалетными, имеющими то же самое решение.

    \textsc{Пример}. Задача нахождения значения функции в точке эквивалентна
    нахождению производной функции в точке.
    
  \item \textit{Методы аппроксимации} (не путать с задачей аппрокиимации).
    Исходная задача заменяется другой, решение которой в некотором смысле близко
    к решению исходной задачи. Вводится количественная мера такой близости,
    которая оценивается и сравнивается с пороговым значением, допустимым для
    конкретной практической задачи.

    \textsc{Пример}. Абсолютное значение температуры.

    Разделяют два подхода к аппроксимации: \textit{линеаризация} --- фрагмент
    кривой заменяется прямой, и \textit{дискретизация} --- непрерывная кривая
    заменяется на конечный набор точек.

  \item \textit{Итерационные методы}. Предполагают вычисление приближения к
    решению в текущий момент времени по значению в предыдущий момент времени. В
    ряде случаев говорят о построении нового приближения к решению по значению
    предыдущего. Метод может быть \textit{сходящимся} и \textit{расходящимся}.
    Необходимо анализировать условие сходимости. Кроме того, необходимо
    определение начального приближения, а также условие окончания счёта.

  \item \textit{Методы Монте-Карло (методы испытаний)}.

    \textsc{Пример}. Метод Монте-Карло для вычисления значения определённого
    интеграла.

\end{enumerate}

\section{Лекция от 19.04.2024}

\subsection{Приближение функций}

Под \textit{приближением функций} понимают описание исходно заданной функции
$f(x)$, её аналога $g(x)$, при этом $|f(x) - g(x)| \to 0$ и / или $f(x_i) =
g(x_i)$ в некоторых точках.

* Пример с интерполяцией функции по точкам. *

Различают
\begin{itemize}
 \item Задачи интерполяции: $f(x) \approx g(x)$, $f(x_i) = g(x_i)$, $x_i$
    ---  некоторые точки, называемые узлами интерполяции. При этом, как правило,
    значения функций между узлами интеполяции отличаются незначительно (с
    точностью до заданного $\varepsilon$. Здесь $x \in [a, b]$.

  \item Задача аппроксимации. $f(x) \approx g(x)$, $x \in [a, b]$;
    $\sum(f(x_i) - g(x_i))^2 \to \min$ --- метод наименьших квадратов.

  \item Задача экстраполяции. $f(x) \approx g(x)$, $x \notin [a, b]$.
\end{itemize}

Отметим, что функции могут задаваться
\begin{enumerate}
  \item Аналитически.

    \textbf{Примеры}.
    \begin{itemize}
      \item $f(x) = e^x$;
      \item $f(x) = \sin(\frac{\pi}{2}x)$, $x \in [0, 4]$;
      \item $f(x) = \int_a^b e^x x^5 dx$, $x \in [a, b]$;
    \end{itemize}

  \item Таблично.

  \item Графически.
\end{enumerate}

\textbf{Пример}. Интерполяция кубическим сплайном.
$S_i(x) = a_i + b_i (x - x_{i - 1}) + c_i(x-x_{i-1}) + d_i(x-x{i-1})^3$,
$i = 1, ... ,n$ --- узел интерполяции.

Условие гладкости функции:
$S_i(x_i) = S_{i+1}(x_i)$,
$S_i'(x_i) = S_{i+1}'(x_i)$,
$S_i''(x_i) = S_{i+1}''(x_i)$,
$i = 1, ... ,n -1$.

Условия гладкости на краях:

...

\section{Вычислительные задачи. Вычислительные алгоритмы.}

Вычислительная задача --- это задача, в основе которой лежит вычислительный
метод (все численные методы, а также точные, реализуемые на компьютере).
Вычислительный метод определяет вычислительный алгоритм.

После выполнения конечного числа элементарных действий ...

\end{document}
